\documentclass[journal,12pt,twocolumn]{IEEEtran}
\usepackage{amsmath}

%
\usepackage{setspace}
\usepackage{gensymb}
%\doublespacing
\singlespacing

\usepackage{graphicx}
%\usepackage{amssymb}
%\usepackage{relsize}
\usepackage[cmex10]{amsmath}
%\usepackage{amsthm}
%\interdisplaylinepenalty=2500
%\savesymbol{iint}
%\usepackage{txfonts}
%\restoresymbol{TXF}{iint}
%\usepackage{wasysym}
\usepackage{amsthm}
%\usepackage{iithtlc}
\usepackage{mathrsfs}
\usepackage{txfonts}
\usepackage{stfloats}
\usepackage{bm}
\usepackage{cite}
\usepackage{cases}
\usepackage{subfig}
%\usepackage{xtab}
\usepackage{longtable}
\usepackage{multirow}
%\usepackage{algorithm}
%\usepackage{algpseudocode}
\usepackage{enumitem}
\usepackage{mathtools}
\usepackage{steinmetz}
\usepackage{tikz}
\usepackage{circuitikz}
\usepackage{verbatim}
\usepackage{tfrupee}
\usepackage[breaklinks=true]{hyperref}
%\usepackage{stmaryrd}
\usepackage{tkz-euclide} % loads  TikZ and tkz-base
%\usetkzobj{all}
\usetikzlibrary{calc,math}
\usepackage{listings}
    \usepackage{color}                                            %%
    \usepackage{array}                                            %%
    \usepackage{longtable}                                        %%
    \usepackage{calc}                                             %%
    \usepackage{multirow}                                         %%
    \usepackage{hhline}                                           %%
    \usepackage{ifthen}                                           %%
  %optionally (for landscape tables embedded in another document): %%
    \usepackage{lscape}     
\usepackage{multicol}
\usepackage{chngcntr}
%\usepackage{enumerate}

%\usepackage{wasysym}
%\newcounter{MYtempeqncnt}
\DeclareMathOperator*{\Res}{Res}
%\renewcommand{\baselinestretch}{2}
\renewcommand\thesection{\arabic{section}}
\renewcommand\thesubsection{\thesection.\arabic{subsection}}
\renewcommand\thesubsubsection{\thesubsection.\arabic{subsubsection}}

\renewcommand\thesectiondis{\arabic{section}}
\renewcommand\thesubsectiondis{\thesectiondis.\arabic{subsection}}
\renewcommand\thesubsubsectiondis{\thesubsectiondis.\arabic{subsubsection}}

% correct bad hyphenation here
\hyphenation{op-tical net-works semi-conduc-tor}
\def\inputGnumericTable{}                                 %%

\lstset{
%language=C,
frame=single, 
breaklines=true,
columns=fullflexible
}
%\lstset{
%language=tex,
%frame=single, 
%breaklines=true
%}

\begin{document}
%


\newtheorem{theorem}{Theorem}[section]
\newtheorem{problem}{Problem}
\newtheorem{proposition}{Proposition}[section]
\newtheorem{lemma}{Lemma}[section]
\newtheorem{corollary}[theorem]{Corollary}
\newtheorem{example}{Example}[section]
\newtheorem{definition}[problem]{Definition}
%\newtheorem{thm}{Theorem}[section] 
%\newtheorem{defn}[thm]{Definition}
%\newtheorem{algorithm}{Algorithm}[section]
%\newtheorem{cor}{Corollary}
\newcommand{\BEQA}{\begin{eqnarray}}
\newcommand{\EEQA}{\end{eqnarray}}
\newcommand{\define}{\stackrel{\triangle}{=}}
\bibliographystyle{IEEEtran}
%\bibliographystyle{ieeetr}
\providecommand{\mbf}{\mathbf}
\providecommand{\pr}[1]{\ensuremath{\Pr\left(#1\right)}}
\providecommand{\qfunc}[1]{\ensuremath{Q\left(#1\right)}}
\providecommand{\sbrak}[1]{\ensuremath{{}\left[#1\right]}}
\providecommand{\lsbrak}[1]{\ensuremath{{}\left[#1\right.}}
\providecommand{\rsbrak}[1]{\ensuremath{{}\left.#1\right]}}
\providecommand{\brak}[1]{\ensuremath{\left(#1\right)}}
\providecommand{\lbrak}[1]{\ensuremath{\left(#1\right.}}
\providecommand{\rbrak}[1]{\ensuremath{\left.#1\right)}}
\providecommand{\cbrak}[1]{\ensuremath{\left\{#1\right\}}}
\providecommand{\lcbrak}[1]{\ensuremath{\left\{#1\right.}}
\providecommand{\rcbrak}[1]{\ensuremath{\left.#1\right\}}}
\theoremstyle{remark}
\newtheorem{rem}{Remark}
\newcommand{\sgn}{\mathop{\mathrm{sgn}}}
\providecommand{\abs}[1]{$\left\vert#1\right\vert$}
\providecommand{\res}[1]{\Res\displaylimits_{#1}} 
\providecommand{\norm}[1]{$\left\lVert#1\right\rVert$}
%\providecommand{\norm}[1]{\lVert#1\rVert}
\providecommand{\mtx}[1]{\mathbf{#1}}
\providecommand{\mean}[1]{E$\left[ #1 \right$]}
\providecommand{\fourier}{\overset{\mathcal{F}}{ \rightleftharpoons}}
%\providecommand{\hilbert}{\overset{\mathcal{H}}{ \rightleftharpoons}}
\providecommand{\system}{\overset{\mathcal{H}}{ \longleftrightarrow}}
    %\newcommand{\solution}[2]{\textbf{Solution:}{#1}}
\newcommand{\solution}{\noindent \textbf{Solution: }}
\newcommand{\cosec}{\,\text{cosec}\,}
\providecommand{\dec}[2]{\ensuremath{\overset{#1}{\underset{#2}{\gtrless}}}}
\newcommand{\myvec}[1]{\ensuremath{\begin{pmatrix}#1\end{pmatrix}}}
\newcommand{\mydet}[1]{\ensuremath{\begin{vmatrix}#1\end{vmatrix}}}
%\numberwithin{equation}{section}
\numberwithin{equation}{subsection}
%\numberwithin{problem}{section}
%\numberwithin{definition}{section}
\makeatletter
\@addtoreset{figure}{problem}
\makeatother
\let\StandardTheFigure\thefigure
\let\vec\mathbf
%\renewcommand{\thefigure}{\theproblem.\arabic{figure}}
\renewcommand{\thefigure}{\theproblem}
%\setlist[enumerate,1]{before=\renewcommand\theequation{\theenumi.\arabic{equation}}
%\counterwithin{equation}{enumi}
%\renewcommand{\theequation}{\arabic{subsection}.\arabic{equation}}
\def\putbox#1#2#3{\makebox[0in][l]{\makebox[#1][l]{}\raisebox{\baselineskip}[0in][0in]{\raisebox{#2}[0in][0in]{#3}}}}
     \def\rightbox#1{\makebox[0in][r]{#1}}
     \def\centbox#1{\makebox[0in]{#1}}
     \def\topbox#1{\raisebox{-\baselineskip}[0in][0in]{#1}}
     \def\midbox#1{\raisebox{-0.5\baselineskip}[0in][0in]{#1}}
\vspace{3cm}
\title{Matrix Theory Assignment 1}
\author{Ankur Aditya: EE20RESCH11010}



\begin{document}



\maketitle
\thispagestyle{empty}
\pagestyle{empty}


%%%%%%%%%%%%%%%%%%%%%%%%%%%%%%%%%%%%%%%%%%%%%%%%%%%%%%%%%%%%%%%%%%%%%%%%%%%%%%%%
\begin{abstract}

This document contains the procedure to get image of a point in a line. 

\end{abstract}
\\
Download the python code from the below link. Go through the README file in the reposotory.
%
\begin{lstlisting}
https://github.com/ankuraditya13/EE5609-Assignment-1
\end{lstlisting}
%
\\
\begin{comment}
and latex-tikz codes from 
%
\begin{lstlisting}
https://github.com/ankuraditya13/EE5609-Assignment-1
\end{lstlisting}
%
\end{comment}
\section{Problem}

%%%%%%%%%%%%%%%%%%%%%%%%%%%%%%%%%%%%%%%%%%%%%%%%%%%%%%%%%%%%%%%%%%%%%%%%%%%%%%%%
\textbf{Find the image of the point} \begin{pmatrix}
    3\\
    8
    \end{pmatrix} \textbf{with respect to the line} 

   \begin{equation}
         \begin{pmatrix}
    1 & 3
    \end{pmatrix} \vec{x} = 7 
\end{equation}       

\section{Solution}


For this problem, I am considering the general case. Let the Equation of line be ax + by = c and let the coordinates of, \\
\textbf{P}(given point) = \begin{pmatrix}
    x1\\
    y1
    \end{pmatrix} \\
    
\textbf{Q}(image point) = \begin{pmatrix}
    x2\\
    y2
    \end{pmatrix} \\ 
    
\textbf{R}(point on mirror) = \begin{pmatrix}
    x3\\
    y3
    \end{pmatrix}
\\

Let vector \textbf{n} = \begin{pmatrix}
    a\\
    b
    \end{pmatrix}
\\ \\
\textbf{Let m be the directional vector along the line} \\ \textbf{ax + by = c} hence, m = \begin{pmatrix}
    b & -a
    \end{pmatrix} \\ 

Let m1 and m2 be the slopes of two prependicular lines,\\ \\
Now, m1 = \(\frac{y2 - y1}{x2 - x1}\) and m2 = \(\frac{-a}{b}\)\\     \\ 
Now for perpendicular lines m1m2 = -1, which in vector can be written as:\\
\begin{equation}
   \vec{m}^T\vec{R} = \vec{m}^T\vec{P}  
\end{equation}

Similarly in vector form line eqauation ax + by + c = 0 is given as,
\begin{equation}
   \vec{n}^T\vec{Q} =c
\end{equation}

\begin{figure}[!ht]
\centering
\includegraphics[scale=0.5]{Figure_1.png}\\
\caption{Image of a point in 2D line}
\label{fig1}
\end{figure}



By property in Figure 0, the line PR bisects the mirror equation perpendicularly. Hence, 
\begin{equation}
    2\vec{Q} = \vec{P} + \vec{R}
\end{equation}

Hence, From the equation (2.0.3) and (2.0.4)
\begin{equation}
    \vec{n}^T\vec{R} = 2c - \vec{n}^T\vec{P}
\end{equation}

Now, form equation (2.0.5) and (2.0.2) we get,

\begin{equation}
    \begin{pmatrix}
    \vec{m} & \vec{n}
    \end{pmatrix} ^T  \vec{R} = \begin{pmatrix}
    \vec{m} & -\vec{n}
    \end{pmatrix} ^T  \vec{P} + \begin{pmatrix}
    0 \\
    2c
    \end{pmatrix} 
\end{equation}
Hence upon solving the equation for point R using the property,
\begin{pmatrix}
    \vec{m} & -\vec{n}
    \end{pmatrix} = \begin{pmatrix}
    \vec{m} & \vec{n}
    \end{pmatrix}  \begin{pmatrix}
    1 & 0\\
    0 & -1 
    \end{pmatrix} \textbf{we get,} \\ 

\\

\begin{equation}
    \(\frac{\vec{R}}{2}\) = \(\frac{\vec{m}\vec{m}^T - \vec{n}\vec{n}^T}{\vec{m}^T\vec{m} + \vec{n}^T  \vec{n}}\)\vec{P} + c  \(\frac{\vec{n}}{||\vec{n}||^2}\)
\end{equation}


     
 Hence, substituting the value of x1 = 3, y1 = 8, a = 1, b = 3 and c = 7 we get,\\ 
 \textbf{P}(given point)  = \begin{pmatrix}
    3 \\ 
    8
    \end{pmatrix} \\ 
 \textbf{m} (direction vector) = \begin{pmatrix}
    3 \\
     -1
    \end{pmatrix} \\ \\
\textbf{n} = \begin{pmatrix}
   1\\
   3
    \end{pmatrix} \\ \\
 Norm, \||\vec{n}\|| = $\sqrt[2]{a^2+b^2}$ \\
 
Substituting these values in equation (2.0.6) we get,\\ 
\begin{equation}
\textbf{R} = \begin{pmatrix}
    -1 \\
    -4
    \end{pmatrix} \\ \\
\end{equation}
 
 Hence, it is the required answer for image of \textbf{P} in line \begin{pmatrix}
    1 & 3
    \end{pmatrix} \textbf{x} =7. 
 
 

 

\end{document}
